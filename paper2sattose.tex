\documentclass[a4paper]{article}
\usepackage{graphicx}
\usepackage{twocolceurws}

\title{Damegender: Writing and Comparing Gender Detection Tools}

\author{
David Arroyo Menéndez \\ GSYC \\ Universidad Rey Juan Carlos I \\ d.arroyome@alumnos.urjc.es
\and
Jesús González Barahona \\ GSYC \\ Universidad Rey Juan Carlos I \\ jgb@gsyc.es
}

\institution{}


\begin{document}
\maketitle

\begin{abstract}
The variable sex (male or female) is one of most used variables for
any study in sociology, but this variable can be hidden in Internet
communities. The gender detection from a name is an important problem
in Natural Language Processing to decide if a string labeled as name
is classified as male or female. An engineer will find useful
make gender detection from a name retrieving information from social
networks, mailing lists, instant messaging, software repositories,
papers, etc. To achieve gender equality and empower all women and
girls is a goal in sustanaible development in United Nations, so to
measure the gender gap is a previous step to find solutions to reduce
it.

Nowadays, there are several Application Programming Interfaces to
guess gender from a name. This kind of software has the database
based on propietary databases and the software is not free, so some
scientific works are difficult to reproduce.

In this paper, we are envisioning how to solve these problems,
offering a solution with a free license and open data names from
official census useful to replace, use and/or compare these apis with
very good results. This tool provides Machine Learning to predict
strings, that's useful to guess diminutives or nicknames.
\end{abstract}


\section{Introduction}

There are different ways to detect gender from a person name and
perhaps a surname: census, wikipedia, self-references in trust
websites, ... The most common way to detect gender from a name is the
Application Programming Interfaces with a good popularity, for
example, genderapi, namsor, genderize, ...

The problems addressed are:
\begin{itemize}
\item Evaluate quality/price with different commercial solutions.
\item Think about solutions using free licenses.
\item Treatment with names without census, for example, nicknames,
  diminutives, ...
\item Massive gender detection from Internet, for example, mailing
  lists, software repositories, ...
\end{itemize}

In this paper, these problems are faced writing a Python solution for:

\begin{itemize}
\item To evaluate quality of different solutions applying metrics
  suggested by ~\cite{10.7717/peerj-cs.156}
\item To understand the current technology in detail, I have developed
  a tool guessing gender from a name giving support to Spanish and
  English from the open data census provides by the states.
\item To fix the problem with nicknames and diminutives, we have
  developed a machine learning solution to strings not found in the
  census dataset.
\item To do proof-of-concept tests applying Perceval to detect
  gender in mailing lists and software repositories.
\end{itemize}


\section{First Level Heading}

First level headings are all flush left, initial caps, bold and in point
size 12. One line space before the first level heading and $1/2$ line
space after the first level heading.

\subsection{Second Level Heading}

Second level headings must be flush left, initial caps, bold and in point
size 10. One line space before the second level heading and $1/2$ line
space after the second level heading.

\subsubsection{Third Level Heading}

Third level headings must be flush left, initial caps and bold.
One line space before the third level heading and $1/2$ line
space after the third level heading.

\paragraph{Fourth Level Heading}

Fourth level headings must be flush left, initial caps and roman type.
One line space before the fourth level heading and $1/2$ line
space after the fourth level heading.

\subsection{Citations In Text}

Citations within the text should indicate the author's last name and
year\cite{Knuth-vol3}. Reference style\cite{Comer-btree}
should follow the style that you are used to using, as long as the
citation style is consistent.

\subsubsection{Footnotes}

Indicate footnotes with a number\footnote{This is a sample footnote} in
the text. Place the footnotes at the bottom of the page they appear on.
Precede the footnote with a vertical rule of 2 inches (12 picas).

\subsubsection{Figures}

All artwork must be centered, neat, clean and legible. Do not use pencil
or hand-drawn artwork. Figure number and caption always appear after the
the figure. Place one line space before the figure, one line space
before the figure caption and one line space after the figure caption.
The figure caption is initial caps and each figure is numbered
consecutively.

Make sure that the figure caption does not get separated from the
figure. Leave extra white space at the bottom of the page to avoid
splitting the figure and figure caption.

Figure \ref{fig1} shows how to include a figure as encapsulated postscript.
The source of the figure is in file {\tt fig1.eps}.

\begin{figure}[ht]
\begin{center}
\includegraphics[height=4cm]{fig1}
\caption{Sample EPS figure }
\label{fig1}
\end{center}
\end{figure}

Below is another figure using LaTeX commands.


\begin{figure}[ht]
\begin{center}
\setlength{\unitlength}{1pt}
\footnotesize
\begin{picture}(160,80)
        \put(0,0){\framebox(160,80)[]{}}
        \put(10,35){\framebox(80,40){}}
        \put(100,20){\framebox(40,20){}}
        \put(70,10){\framebox(20,10){}}
        \put(20,5){\framebox(10,5){}}
\end{picture}
\caption{Sample Figure Caption}
\end{center}
\end{figure}

\subsubsection{Tables}

All tables must be centered, neat, clean and legible. Do not use pencil
or hand-drawn tables. Table number and title always appear before the
table.

One line space before the table title, one line space after the table
title and one line space after the table. The table title must be
initial caps and each table numbered consecutively.

\begin{table}[ht]
\begin{center}
\caption{Sample Table}

\bigskip

\begin{tabular}{|l|l|r|}
\hline
A & B & 1\\ \hline
C & D & 2\\
E & F & 3\\ \hline
\end{tabular}
\end{center}
\end{table}


\subsubsection{Handling References}

Use a first level heading for the references. References follow the
acknowledgements.


\subsubsection{Acknowledgements}

Use a third level heading for the acknowledgements. All acknowledgements
go at the end of the paper.




%\bibliographystyle{alpha} 
%\bibliography{samplebib}
%inline the .bbl file directly for mailing to authors.

\begin{thebibliography}
  
\bibitem[Com79]{Comer-btree}
D.~Comer.
\newblock The ubiquitous b-tree.
\newblock {\em Computing Surveys}, 11(2):121--137, June 1979.

\bibitem[Knu73]{Knuth-vol3}
Santamaría, Lucía and Mihaljević, Helena
\newblock {\em Comparison and benchmark of name-to-gender inference services}.
\newblock Addison-Wesley, 1973.

\end{thebibliography}

\end{document}


