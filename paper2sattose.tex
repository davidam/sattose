\documentclass[a4paper]{article}
\usepackage{graphicx}
\usepackage{twocolceurws}

\title{Damegender: \\ Writing and Comparing Gender Detection Tools}

\author{
David Arroyo Men\'endez \\ GSyC \\ Universidad Rey Juan Carlos \\ d.arroyome@alumnos.urjc.es
\and
Jesus Gonzalez-Barahona \\ GSyC \\ Universidad Rey Juan Carlos \\ jgb@gsyc.es
}

\institution{}


\begin{document}
\maketitle

\begin{abstract}
The variable sex (male or female) is one of most used variables for
any study in sociology, but this variable can be hidden in Internet
communities. The gender detection from a name is an important problem
in Natural Language Processing to decide if a string labeled as name
is classified as male or female. An engineer will find useful
make gender detection from a name retrieving information from social
networks, mailing lists, instant messaging, software repositories,
papers, etc. To achieve gender equality and empower all women and
girls is a goal in sustainable development in United Nations, so to
measure the gender gap is a previous step to find solutions to reduce
it.

Nowadays, there are several Application Programming Interfaces to
guess gender from a name. This kind of software has the database
based on proprietary databases and the software is not free, so some
scientific works are difficult to reproduce.

In this paper, we are envisioning how to solve these problems,
offering a solution with a free license and open data names from
official census useful to replace, use and/or compare these APIs with
very good results. This tool provides Machine Learning to predict
strings, that is useful to guess diminutives or nicknames.
\end{abstract}


\section{Introduction}

There are different ways to detect gender from a person name and
perhaps a surname: census, Wikipedia, self-references in trust
websites, \ldots The most common way to detect gender from a name is the
Application Programming Interfaces with a good popularity, for
example, genderapi, namsor, genderize, \ldots

The problems addressed are:
\begin{itemize}
\item Evaluate quality/price with different commercial solutions.
\item Think about solutions using free licenses.
\item Treatment with names without census, for example, nicknames,
  diminutives, \ldots
\item Massive gender detection from Internet, for example, mailing
  lists, software repositories, \ldots
\end{itemize}

In this paper, these problems are faced writing a Python solution for:

\begin{itemize}
\item To evaluate quality of different solutions applying metrics
  suggested by Santamaría and Mihaljević~\cite{10.7717/peerj-cs.156}
\item To understand the current technology in detail, I have developed
  a tool guessing gender from a name giving support to Spanish and
  English from the open data census provides by the states.
\item To fix the problem with nicknames and diminutives, we have
  developed a machine learning solution to strings not found in the
  census dataset.
\item To do proof-of-concept tests applying Perceval~\cite{duenas2018perceval} to detect
  gender in mailing lists and software repositories.
\end{itemize}

\section*{Acknowledgements}

Here the acknowledgements come...

\bibliographystyle{alpha} 
\bibliography{bibtex}

\end{document}


